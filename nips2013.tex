\documentclass{article} % For LaTeX2e
\usepackage{nips13submit_e,times}
\usepackage{hyperref}
\usepackage{url}
\usepackage{graphicx}
\usepackage{subcaption}
%\documentstyle[nips13submit_09,times,art10]{article} % For LaTeX 2.09


\title{CSC 2515 Project: SVHN}


\author{
Didier Landry\\
Department of Electrical and Computer Engineering\\
University of Toronto\\
\texttt{didier.landry@mail.utoronto.ca} \\
\And
Dustin Kut Moy Cheung\\
Department of Electrical and Computer Engineering\\
University of Toronto\\
\texttt{dustin.kutmoycheung@mail.utoronto.ca} \\
}

% The \author macro works with any number of authors. There are two commands
% used to separate the names and addresses of multiple authors: \And and \AND.
%
% Using \And between authors leaves it to \LaTeX{} to determine where to break
% the lines. Using \AND forces a linebreak at that point. So, if \LaTeX{}
% puts 3 of 4 authors names on the first line, and the last on the second
% line, try using \AND instead of \And before the third author name.

\newcommand{\fix}{\marginpar{FIX}}
\newcommand{\new}{\marginpar{NEW}}

\nipsfinalcopy % Uncomment for camera-ready version

\begin{document}


\maketitle

\begin{abstract}
Detecting and reading text from photographs is a challenging computer vision problem that has garnered a lot of work in recent years. Being able to accurately localize, and recognize arbitrary digits from natural images is hard due to the complex scenes in those images. In this report, we apply convolutional neural networks to learn a unique set of features optimized for this task, and discuss the evolution of our neural network topology to achieve a high accuracy on the validation set. We use over 500,000 labeled digits obtained from the SVHN\cite{svhn} dataset for training. Moreover, we also describe our attempt to localize digits from unconstrained images by using image processing techniques and unsupervised learning.
\end{abstract}

\section{Background}
Convolutional neural networks\cite{lecun_convolutional_svhn} are biologically-inspired neural networks that uses identical copies of the same neuron for training, allowing the network to have a large number of units while keeping the number of parameters of those neurons small. The individual neurons are generally connected to filtered overlapping regions of the image. The convolutional layers are connected to pooling layers to reduce dimensionality. One popular choice for this kind of layer is the max-pooling layer, which extracts the maximum of features over small blocks of the previous layer.

Generally 2D convolutional neural networks are used in computer vision to learn features for extraction information. The 2D layer will look at patches of images to generate features such as the detection of the presence of an edge, or a particular texture.

The LeNet models are a popular family of models used in computer vision. They consist of multiple layers of convolutional layers and max-pooling, followed by fully-connected neural networks. We will be using one of those LeNet network for our training.

Recently, convolutional neural networks have been applied with dropout training in computer vision. When training deep neural networks with a small training set, the training generally leads to overfitting. One method to fix this is to simply stop training when the performance on the validation set starts to get worse. Another popular technique is to use L2 regularization on the neurons’ weights. Dropout is yet another technique used to prevent overfitting by randomly dropping out units in a neural network.

The Street View House Numbers(SVHN)\cite{svhn} dataset is a popular training set consisting of Street View images cropped to either show single digits (Format 2) and multiple digits (Format 1). Those images are extracted using a combination of automated algorithms and Amazon Mechanical Turk (AMT) framework.


\section{Problem Description}
The project consists of classifying digits from street view images. The SVHN dataset will be used for training. All the images in the Format 2 dataset have a fixed 32x32 resolution with a digit centered at the image. There are ten classes in total. The images show vast intra-class  variations due to image distortions that happen in natural scene pictures. Factors that cause those image distortions include lighting, shadows, motion, and focus blurs.

The Format 1 dataset consists of images with different resolutions with multiple digits in each image. The images are not well cropped and not centered. The task is to first detect the digits with a bounding box, then classify each digit. The images in the Format 1 dataset also show the same image variations as in the Format2 dataset.

To facilitate the process of implementing and debugging the digit recognition, the Format 2 data was used to test the digit recognition neural network. The digit segmentation was treated as a different problem because it (obviously) degrades the quality of the input images fed to the neural network. However, the digit segmentation algorithm produces 32x32 images that could, then, be fed to the neural network for recognition.

\section{Process}
\subsection{Digit Recognition}
\subsubsection{Frameworks Used}
Throughout the project, we used the popular Theano project\cite{theano}, together with Lasagne\cite{lasagne}, numpy, scipy\cite{numpyscipy}, and NoLearn\cite{nolearn} libraries. Most of our experiments are executed using an EC2 machine with a GPU to exploit the speed gains of training our Neural Network with the CUDA framework.

\subsubsection{Basic Neural Network}
Our first approach is to train a basic Neural Network with two hidden layers with the grayscale images of the train dataset. The nonlinearity activation function used for our hidden layers is a Rectifier, whereas the output layer uses a softmax layer. The grayscale images are normalized before being used as input. The output layer consists of 10 units, each representing a particular digit. The guessed digit from the Neural Network is determined by the unit with the largest value.

\subsubsection{Convolutional Neural Network (CNN)}
Another more successful approach that we attempted is to add convolutional neural networks to our basic neural network. We added 2 convolutional and max-pooling layers in between the inputs and the ReLu hidden layers.


\subsubsection{Improvement to CNN}
One of the issues that can be experienced while using deep neural networks is overfitting, where the neural networks overtrain on the training set, while the validation set gets worse. One of the methods to rectify overlearning is by the use of Dropout layers, with a dropout probability of 0.5.

Another improvement to our Convolutional Neural Network is to employ pre-processing on our images to boost some features that could help our network train faster and better. One of the popular techniques is the Local Contrast Normalization, which is applied to each of the three color channels of the images. It has been used extensively in Neural Networks and huge improvements on the training part has been observed.

\subsubsection{Augmenting the training set}
Another way to decrease overfitting in our neural network is to simply augment our training set. To achieve this, we use the extra set (~500k images) available from the SVHN dataset to augment our initial training set of ~ 73k images. Another way that we could have augmented our dataset would be to add noise and slightly rotate our existing training set to produce more data. 

\subsection{Image Segmentation}
The unsegmented data set contains images of houses’ address plates at different levels of zoom of zoom and isolation. Moreover, the number of digits that a given image contains is also unknown. Consequently, separating each individual digit is far from trivial.

One approach could have been to randomly take squares from the image and train a classifier to decide whether it contains a digit or not. This could have worked rather well if the goal was just to find digits in an image. However, for segmentation (one image per digit, centered), its performance would have been random at best. Since, to properly decode an address, the information contained in the number of digits and their order is as important as the digits themselves, this approach was rejected.

Accordingly, here, we wrote an algorithm loosely inspired on MATLAB ocr example. It can be found attached to this document.

First, the images are preprocessed: they are resized to a height of 200 pixels while keeping the aspect ratio the same and converted to grayscale.

\begin{figure}[!htb]
\minipage{0.32\textwidth}
  \includegraphics[width=\linewidth]{images/image02}
  \caption{Original Image (2200.png in training set)}
  \label{fig:orig}
\endminipage\hfill
\minipage{0.32\textwidth}
  \includegraphics[width=\linewidth]{images/image06}
  \caption{Vertical Edges (Sobel vert. edge detection)}
  \label{fig:vertical}
\endminipage\hfill
\minipage{0.32\textwidth}%
  \includegraphics[width=\linewidth]{images/image00}
  \caption{Mask obtained by morphological closing, after filtering}
  \label{fig:mask}
\endminipage
\end{figure}

The second step is to find the areas of the image that have a high probability of containing text (an approach similar to that of A. Coates et. al.’s “detection”). For that, we apply a sobel vertical edge detection algorithm and, then, morphological closing on the found edges (very scattered) to create connected regions. This worked well because the vertical boundaries in numbers tend to create a rectangular blob of points that, when morphologically closed, completely contains the digit %(Figs.\ref{fig:vertical} \& \ref{fig:mask}). Vertical edges such as the wall, in this example, get filtered easily because, after closing, they create really narrow regions. 

Depending on the image, this preliminary analysis can produce a lot of possibilities. To solve this problem, the candidate connected zones are filtered by size, aspect ratio and solidity. Areas that are too narrow, fat or have a low solidity (level of vertical gradients in a box containing the connected zone) are discarded. The zone that has the greatest probability of containing a digit is then passed to the next step.

\begin{figure}[!htb]
\minipage{0.45\textwidth}
  \includegraphics[width=\linewidth]{images/image08}
  \caption{Digit splitting based on # of white pixels per column. The red line is the threshold.}
  \label{fig:thres}
\endminipage\hfill
\minipage{0.45\textwidth}
  \includegraphics[width=\linewidth]{images/image04}
  \caption{Split Zones. Leftmost zone was discarded because of its small width}
  \label{fig:split}
\endminipage\hfill
\end{figure}

The third step is to detect the number of digits present in the previously found area. For that, the image is turned into binary (black and white) using Otsu’s [thresholding] method . Then, the after inverting the image if the background was white, the number of white pixels in each column (assumed to be the digit) is counted. If it falls below a threshold (that is a function of the maximum number of white pixels in the image’s column), a split point is saved (see Fig\ref{fig:thres} \& \ref{fig:split}).


\begin{figure}
\begin{subfigure}{.3\textwidth}
  \centering
  \includegraphics[width=.8\linewidth]{images/image03}
\end{subfigure}%
\begin{subfigure}{.3\textwidth}
  \centering
  \includegraphics[width=.8\linewidth]{images/image07}
\end{subfigure}
\begin{subfigure}{.3\textwidth}
  \centering
  \includegraphics[width=.8\linewidth]{images/image01}
\end{subfigure}
\caption{Final Output}
\label{fig:output}
\end{figure}

Finally, the split zones are, again, filtered by width and cropped into 32x32 pixels images (extending laterally to get a 1:1 aspect ratio, if necessary) (Fig\ref{fig:split} \& \ref{fig:output}).



\section{Experimental Results}
\subsection{Digit Recognition}
\subsection{Image Segmentation}

It is important to point out that the digit segmentation algorithm is mostly an engineered solution (in that it doesn’t learn from the data); it just applies decisions based on hard-coded thresholds, gradients and masking.

This is because this engineered solution systematically performed better than the machine learning techniques we used and that any attempt to integrate machine learning algorithms degraded the results. Below are a few of the techniques that were tried but later abandoned.

\subsubsection{Better Boundaries}
An attempt to use a more advanced edge detection algorithm, P. Isola et. al.’s crisp boundary detection\cite{isola2014crisp}, was made without success. Indeed, low contrast pictures, it would detect tiles due to colour imperfections in the image file’s encoding while, in images where a strong colour contrast that is not due to the address’ digits is present (such as the edge of a wall), it would completely disregard the lower contrast changes due to the digits.

\subsubsection{PCA for Segmentation}
In order to make the separation of the digits easier, an attempt was made to use PCA to generate the histogram in Fig. 4 (and to rectify the image when the address was diagonal). This, for instance, would have made it easier to separate numbers that are placed in diagonal because it would have been rotated to an horizontal orientation (the current algorithm tends to fail at this task because of the vertical overlap between the numbers). However, this approach was abandoned because it also had major adverse effects, such as a tendency to flip the image vertically or horizontally that made it perform worse than a simple count of white pixels.

\subsubsection{kNN for Segmentation}
Further, kNN was tried as a way to locate the centers of the digits. The white pixel count was be used to guess a value for k and, then, the image was split using the centroids found by kNN on the thresholded image. This caused more errors than a simple split however, because, even if the number of digits found was correct, the centroids were sometimes distributed vertically (i.e. two centroids on the same number) instead of all horizontally. Forcing the kNN to be only horizontal produced slightly better results but was still not as good as a simple split because it was sensitive to noise (address plate edges) and weight distributions in the digits (a “5” typically has a centroid in its upper-left portion, for instance).

\subsubsection{Results}
The segmentation results are reported below. Please note that the accuracy is based on a “success” being defined as properly isolating and centering ALL the digits in an image (not more or less). For instance, the case where all the digits of an address were found properly where but the edge of the number plate was mistaken for an “1” would be considered a “fail”. 

Also, since this definition of success is too variable for a computer to properly evaluate it, the “successes” had to be counted manually. This, unfortunately, introduces a bias in the numbers below as it limits the number of examples that can be used to compile the accuracy statistics.

\begin{center}
  \begin{tabular}{ | l || c ||}
    \hline
    Examples (treated with this algorithm) & Accuracy (hand-counted) \% \\ \hline \hline
    100 examples from training set (2200 to 2300.png) &  36 \\ \hline
    100 examples from extras set (3000 to 3100.png) & 37 \\
    \hline
  \end{tabular}
\end{center}

\section{Discussion and Conclusion}
\subsection{Scaling with Millions of Examples}
Scaling our current convolutional neural network to millions of examples is incredibly challenging. Luckily, in lasagne, a lot of optimizations happen in the background. For instance, the batch size is selected automatically depending on the number of examples.

Even though Lasagne and Theano ultimately produces CUDA code that can be run on a GPU, there are further optimizations that could be achieved by using a pure CUDA approach and getting rid of Python dependencies along the way. 

One way to scale our training is to simply use a distributed neural network spanning across several machines. At this amount of data, the overhead involved with communication between the different machines become negligible compared to the speed gain in training in parallel. An example of such a distributed network is DistBelief, developed at Google.

Another way is to simply decrease the pixels of each image used to a smaller size. By having less pixels to train on, the training speed will increase, albeit at the expense of losing some information.

\subsection{Digit Recognition}
Scaling our current convolutional neural network to millions of examples is incredibly challenging. Luckily, in lasagne, a lot of optimizations happen in the background. For instance, the batch size is selected automatically depending on the number of examples.

Even though Lasagne and Theano ultimately produces CUDA code that can be run on a GPU, there are further optimizations that could be achieved by using a pure CUDA approach and getting rid of Python dependencies along the way. 

One way to scale our training is to simply use a distributed neural network spanning across several machines. At this amount of data, the overhead involved with communication between the different machines become negligible compared to the speed gain in training in parallel. An example of such a distributed network is DistBelief, developed at Google.

Another way is to simply decrease the pixels of each image used to a smaller size. By having less pixels to train on, the training speed will increase, albeit at the expense of losing some information.

\subsection{Image Segmentation}
The image segmentation algorithm could be improved to allow for more than one zone of high probability to contain a digit. Currently, if the numbers of the address are too far apart in the image, they get “boxed” separately and, in the end, only one number gets isolated properly. This is because, after the morphological closing of the vertical gradients, all the numbers are assumed to be in the same connected zone (this was a simplifying assumption that proved to be correct in most cases).

It would also be interesting to continue to investigate to find machine learning techniques that could exploit some information that comes with the Format 1 images; the boxes. Indeed, each image comes with the location and size of boxes that surround the each digit. Perhaps, using the number of boxes, a classifier could be trained in a supervised manner to find the number of digits in each image. 

\bibliographystyle{plain}
\bibliography{ml}
\end{document}
